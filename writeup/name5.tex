\documentclass[11pt,a4paper]{article}
\usepackage{times,latexsym}
\usepackage{url}
\usepackage[T1]{fontenc}

%% Package options:
%% Short version: "hyperref" and "submission" are the defaults.
%% More verbose version:
%% Most compact command to produce a submission version with hyperref enabled
%%    \usepackage[]{tacl2018v2}
%% Most compact command to produce a "camera-ready" version
%%    \usepackage[acceptedWithA]{tacl2018v2}
%% Most compact command to produce a double-spaced copy-editor's version
%%    \usepackage[acceptedWithA,copyedit]{tacl2018v2}
%
%% If you need to disable hyperref in any of the above settings (see Section
%% "LaTeX files") in the TACL instructions), add ",nohyperref" in the square
%% brackets. (The comma is a delimiter in case there are multiple options specified.)

\usepackage[]{tacl2018v2}



\renewcommand{\baselinestretch}{0.985}



%\setlength\titlebox{5cm}
% You can expand the titlebox if you need extra space
% to show all the authors. Please do not make the titlebox
% smaller than 5cm (the original size); we will check this
% in the camera-ready version and ask you to change it back.

\newcommand\BibTeX{B{\sc ib}\TeX}
\newcommand\confname{EMNLP-IJCNLP 2019}
\newcommand\conforg{SIGDAT}

% Use the lineno option to display guide line numbers if required.

\usepackage{amsmath}
\usepackage{tikz-dependency}
\DeclareMathOperator*{\argmax}{arg\,max}
\DeclareMathOperator*{\argmin}{arg\,min}
\DeclareMathOperator{\E}{\mathop{\mathbb{E}}}

\usepackage{amssymb}% http://ctan.org/pkg/amssymb
\usepackage{pifont}% http://ctan.org/pkg/pifont
\newcommand{\cmark}{\ding{51}}%
\newcommand{\xmark}{\ding{55}}%


\newcommand{\Prob}{\mathbb{P}}%

%\usepackage{pslatex}
%\usepackage{latexsym}
\usepackage[english]{babel}
\usepackage[utf8]{inputenc}
\usepackage{bm}
\usepackage{graphicx}
\usepackage{tikz}
\usepackage{xcolor}
\usepackage{url}
%\usepackage[colorinlistoftodos]{todonotes}
\usepackage{rotating}
\usepackage{multirow}





\usepackage[T1]{fontenc}

\usepackage{pslatex}
%\usepackage{latexsym}
\usepackage[english]{babel}
\usepackage[utf8]{inputenc}
\usepackage{amsmath}
\usepackage{bm}
\usepackage{graphicx}
\usepackage{tikz}
\usepackage{xcolor}
\usepackage{url}
%\usepackage[colorinlistoftodos]{todonotes}
\usepackage{rotating}
%\usepackage{natbib}
\usepackage{amssymb}


\usepackage{amsthm}
 

\allowdisplaybreaks

\newcounter{theorem}
\newtheorem{proposition}[theorem]{Proposition}
\newtheorem{corollary}[theorem]{Corollary}
\newtheorem{question}[theorem]{Question}
\newtheorem{example}[theorem]{Example}
\newtheorem{defin}[theorem]{Definition}
\newtheorem{remark}[theorem]{Remark}
\newtheorem{lemma}[theorem]{Lemma}
\newtheorem{thm}[theorem]{Theorem}


\newcommand{\R}[0]{\mathbb{R}}
\newcommand{\Ff}[0]{\mathcal{F}}
\newcommand{\key}[1]{\textbf{#1}}


\newcommand{\soft}[1]{}
\newcommand{\nopreview}[1]{}
\newcommand\comment[1]{{\color{red}#1}}
\newcommand\mhahn[1]{{\color{red}(#1)}}
\newcommand{\rljf}[1]{{\color{blue}[rljf: #1]}}

\newcommand{\thetad}[0]{{\theta_d}}
\newcommand{\thetal}[0]{{\theta_{LM}}}
\newcommand{\thetap}[0]{{\theta_{P}}}


\title{Theoretical Limitations of Self-Attention in Neural Sequence Models}
%\author{Michael Hahn \\ Stanford University \\ mhahn2@stanford.edu}
\date{\today}
\begin{document}
\maketitle
\begin{abstract}
Transformers are emerging as the new workhorse of NLP, showing great success across tasks.
Unlike LSTMs, transformers process input sequences entirely through self-attention.
Previous work has suggested that the computational capabilities of self-attention to process hierarchical structures are limited.
In this work, we mathematically investigate the computational power of self-attention to model formal languages.
Across both soft and hard attention, we show strong theoretical limitations of the computational abilities of self-attention, finding that it cannot model periodic finite-state languages, nor hierarchical structure, unless the number of layers or heads increases with input length.
These limitations seem surprising given the practical success of self-attention and the prominent role assigned to hierarchical structure in linguistics, suggesting that natural language can be approximated well with models that are too weak for the formal languages typically assumed in theoretical linguistics. %limited modeling the formal languages typically assumed in theoretical linguistics.
%Our results precisely describe theoretical limitations of the techniques underlying recent advances in NLP.
\end{abstract}


Transformers are emerging as the new workhorse of NLP, achieving the state-of-the-art in tasks such as language modeling, machine translation, and creating pretrained contextualized word embeddings.
%This poses the question of what kinds of structures they are capable of expressing, and what kinds of computations they are able to perform.
Eschewing recurrent computations, transformers are entirely based on self-attention, performing their computations largely in parallel.
This enables them to scale to very long sequences \cite{vaswani2017attention,dai2019transformer,child2019generating}.
On the other hand, it has been suggested that this limits their expressiveness, as they cannot process input sequentially \cite{tran2018importance,dehghani2018universal,shen2018disan,chen2018best,hao2019modeling}.
One aspect thought to be challenging for sequence models is hierarchical structure and recursion.
Hierarchical structure is widely thought to be essential to modeling natural language, in particular its syntax~\cite{everaert2015structures}.
Consequently, many researchers have studied the capability of recurrent neural network models to capture context-free languages (e.g., \citet{kalinke1998computation,gers2001lstm,gruning2006stack,weiss2018practical,sennhauser2018evaluating,korsky2019computational}) and linguistic phenomena involving hierarchical structure (e.g., \citet{linzen2016assessing,gulordava2018colorless}).
Some experimental evidence suggests that transformers might not be as strong as LSTMs at modeling hierarchical structure~\cite{tran2018importance}, though analysis studies have shown that transformer-based models encode a good amount of syntactic knowledge (e.g., \citet{clark2019bert,lin2019open,tenney2019bert}).

In this work, we examine these questions from a theoretical perspective, asking whether models entirely based on self-attention are theoretically capable of modeling hierarchical structures involving unbounded recursion.
Formally, we study their ability to perform two computations that are thought to be essential to hierarchical structure:
First, their ability to correctly \key{close brackets}, a basic problem underlying all nonregular context-free languages and formalized by the \key{\textsc{Dyck}} language \cite{chomsky1963algebraic}.
Second, their ability to \key{evaluate iterated negation}, a basic component of the task of evaluating logical formulas, amounting to evaluating the \key{\textsc{Parity}} of bitstrings.
We show that neither of these problems can be solved by transformers and similar models relying entirely on self-attention, unless the number or size of parameters increases with the input length.
Besides representing basic building blocks of hierarchical structure, these languages also represent large classes of regular and context-free languages, meaning that our results carry over to classes of other formal languages.
Our results therefore also yield more generally limitations on the ability of self-attention to model finite-state languages and context-free languages.
%Our results entail show strong theoretical limitations of self-attention:
%Transformers essentially cannot model non-counter-free regular languages or languages containing unbounded recursion.

While theoretical work has investigated the power of recurrent neural networks in depth (e.g., \citet{siegelman1991neural, bengio1994learning, weiss2018practical, miller2018recurrent, merrill2019sequential}), the theoretical study of self-attention has begun only recently \citep{perez2019turing,hsieh2019robustness}.
%While computational limitations of transformers have been conjectured to exist~\cite{dehghani2018universal}, 
Our study provides the first theoretical results on limitations in the power of self-attention.
We will provide results both for hard and soft attention settings, using different proof methods.
Our results are strongest in the hard attention setting, holding without further assumptions on activation functions and parameter norms.
In the soft attention settings, we still obtain results assuming smoothness of activation functions as used in practical implementations.

After discussing related work (Section~\ref{sec:related}), we introduce self-attention (Section~\ref{sec:def-selfatt}) and two fundamental formal languages representing regular and context-free languages (Section~\ref{sec:langs}).
We then prove that self-attention cannot model these languages using either hard (Section~\ref{sec:hard}) or soft (Section~\ref{sec:soft}) attention.
Finally, we discuss our results (Section~\ref{sec:discussion}).

\section{Related Work}\label{sec:related}
\paragraph{Prior Work on Self-Attention}
Transformers were proposed by \citet{vaswani2017attention}, previous related work using self-attention includes \citet{cheng2016long,parikh2016decomposable,paulus2017deep,lin2017structured}.
It has been a recurrent suggestion in the literature that transformers, relying entirely on self-attention, are restricted computationally, as they cannot process their input sequentially.
\citet{dehghani2018universal} suggested that %transformers are restricted computationally, motivating their proposal of a variant with adaptive and unbounded number of layers.
%They argued that
transformers cannot compute functions that require sequential processing of input, without providing further details or proofs.
Similarly, \citet{shen2018disan,chen2018best,hao2019modeling} have introduced extensions of transformers with recurrence, citing similar intuitions about limitations of transformers.
Our results provide the first explicit formalization of these limitations.

A few studies have experimentally tested the abilities of transformers to learn structures.
Most related to our work, \citet{tran2018importance} compared the ability of transformers and LSTMs to learn hierarchical structure, specifically English subject-verb agreeement and evaluating logical formulas.
Their experimental results suggested that LSTMs are better at learning hierarchical structure.
\citet{yang2019assessing} experimentally investigated the power of self-attention to extract word order information, finding differences between recurrent and self-attention models; however, these were modulated by the training objective.
\citet{lin2019open} and \citet{tenney2019bert} show that BERT \cite{devlin2018bert} encodes syntactic information.

Theoretical study of transformers was initiated by \citet{perez2019turing}, who theoretically studied the ability of Seq2Seq transformers to emulate the computation of Turing machines.
%There is no contrast between their results and ours:
While we consider incremental modeling of sequences, where the number of computation steps is bounded by the input length $n$, they study the setting in which the transformer computes an unbounded number of autoregressive decoding steps, not bounded in the input length $n$.
Even more recently, and more closely related to our interest here, \citet{hsieh2019robustness} studied the adversarial robustness of transformers.
While they focused on experiments on NLP tasks, they also provided a theoretical analysis, showing that a single self-attention layer with a single head will be robust against input perturbations, assuming that input embeddings are drawn uniformly from the unit sphere.
One of our results, Lemma~\ref{lem:soft-tech}, can be seen as considerably widening the scope of their result, both by avoiding distributional assumptions, and by applying to transformers with arbitrary numbers of heads and layers.


\paragraph{Investigating the Power of Sequence Modeling Architectures}
The computational power of recurrent neural networks has been a focus of study.
A particular focus has been on their ability to learn non-regular context-free languages, thought to provide simple models of recursion and hierarchical structure as found in natural language.

A range of studies has experimentally examined the ability of recurrent networks to model counter languages such as $a^nb^n$ \cite{kalinke1998computation,gers2001lstm,cartling2008implicit,weiss2018practical,suzgun2019evaluating}.
Other work has experimentally studied the performance of reccurent architectures on learning to recognize well-bracketed strings, a similar but more challenging problem \cite{sennhauser2018evaluating,skachkova2018closing,bernardy2018can}.
Beyond modeling formal languages, another line of work has studied the ability of LSTMs to model hierarchical structure as occurring in realistic natural language data \cite{linzen2016assessing,gulordava2018colorless}.

Recently, \citet{merrill2019sequential} and \citet{korsky2019computational} theoretically studied several types of recurrent networks. \citet{merrill2019sequential} showed that -- in the finite precision setting -- LSTMs recognize a subset of the counter languages, whereas GRUs and simple RNNs recognize regular languages.
\citet{korsky2019computational} showed, among other results, that arbitrary-precision RNNs can emulate pushdown automata, and can therefore recognize all deterministic context-free languages.
%Inter alia, their results entail that LSTMs cannot model the multi-symbol Dyck language in the finite precision setting.


A related, though different, strand of research has investigated the power of neural networks to model Turing machines.
A classical result~\cite{siegelman1991neural} states that -- given unlimited computation time -- recurrent networks can emulate the computation of Turing machines.
Very recently, \citet{perez2019turing} have shown the same result for both (argmax-attention) Transformers and Neural GPUs.
The crucial difference between these studies and studies of language recognition is that, in these studies, the networks are allowed to perform unbounded recurrent computations, arbitrarily longer than the input length.

\section{Self-Attention}\label{sec:def-selfatt}
Here we define self-attention as used in Transformers, following \citet{vaswani2017attention}.
We have an input ${\bf x} = x_1 \dots x_n$, where all $x_i$ come from some finite alphabet $\mathcal{V}$, and $x_n$ is an end-of-sequence symbol.
This input is then encoded into a sequence of input embeddings $v_1,\dots,v_n$ using some embedding map $\mathcal{V} \rightarrow \mathbb{R}^k$. %, where $v_i \in \mathbb{R}^k$ are input embeddings, and we assume that $x_n$ encodes an end-of-sequence symbol.
We furthermore have a sequence $p_1, p_2, \dots$ of positional embeddings $p_i \in \mathbb{R}^k$. These are independent of the input ${\bf x}$, and can be computed through some predefined scheme, or could be learned for each position occurring in the training data \citep{vaswani2017attention}.
Input and positional embeddings are combined (e.g., via addition or concatenation) to vectors $y_i^{(0)} = f(v_i, p_i)$ ($i=1, \dots, n$), which we will refer to as Layer 0.

A transformer has a fixed number $L$ of \key{layers}; the \key{activations} $y_i^{(k)}$ at position $i$ of the $k$-th layer ($k=1, \dots, L$) are defined as follows.
Each layer has a set of $H$ \key{attention heads}; we first compute attention scores for the $h$-th head:
\begin{equation}
    a_{i,j}^{(k,h)} = f^{att}_{k,h}\left(y_i^{(k-1)}, y_j^{(k-1)}\right)
\end{equation}
where $f^{att}_{k,h}$ combines the activations from the previous level into an attention score.
This can be implemented e.g. using dot product or additive attention.
The  activation of the head is computed by weighting according to attention weights:
\begin{equation}
    b_{i,k,h} = \sum_{j=1}^n \hat{a}_{i,j}^{(k,h)} y_j^{(k-1)}
\end{equation}
In the soft attention version, the weights $\hat{a}_{i,\cdot}$ are obtained by the softmax operation: $\hat{a}_{i,\cdot} = \operatorname{softmax}(a_{i,\cdot})$.
In the hard attention variant \cite{perez2019turing}, one takes the actual maximum attention values:
$\hat{a}_{i,j}^{(k,h)} = \delta_{j, \argmax_{j'} a_{i,j'}^{(k,h)}}$.\footnote{When there are multiple positions with maximal attention weight; we will assume that the one occurring first in the sequence is chosen. Our analysis also works under other schemes of resolving ties, such as random selection.}
The  per-position activations are then computed as
\begin{equation}
    y_i^{(k)} := f^{act}(y_i^{(k-1)}, b_{i,k,1}, \dots, b_{i,k,H})
\end{equation}
where $f^{act}$ is implemented as a fully-connected feedforward network with a skip-connection \cite{vaswani2017attention}.

\paragraph{Hard and Soft Attention}
There is a choice between soft attention and hard attention \cite{shen2018reinforced,perez2019turing}.
The one prior theoretical study of transformers~\cite{perez2019turing} assumes hard attention.
In practice, soft attention is easier to train with gradient descent; however, analysis studies suggest that attention often concentrates on one or a few positions in trained transformer models \cite{voita2019analyzing,clark2019bert} and that the most important heads are those that clearly focus on a few positions~\cite{voita2019analyzing}, suggesting that attention often behaves  like hard attention in practice. %, and analysis of neural network attention focuses on the positions where attention concentrates
We will  examine both hard (Section~\ref{sec:hard}) and soft (Section~\ref{sec:soft}) attention.
%We will see that, if arbitrary activation functions are allowed, soft attention is significantly more powerful than hard attention.
%On the other hand, in the realistic setting where activation functions obey reasonable regularity conditions (satisfied by ReLU, Tanh, \dots), hard and soft attention result in similar limitations.


\paragraph{Formalizing Language Recognition}
We consider the problem of language recognition, the task of classifying input strings as belonging to or not belonging to a formal language.
Following~\citet{weiss2018practical}, we formalize this as the sequence-to-sequence task of mapping words to labels $1$ (`in the language') and $0$ (`not in the language').
Following the construction of transformers in sequence-to-sequence tasks \cite{vaswani2017attention}, we compute a softmax probability vector for this label from the last activation $y_{n}^{(L)}$, obtained after reading the end-of-sequence symbol.

%We consider the problem of language recognition as in Weiss et al: a Seq2Seq model is tasked with transducing words in the language to $1$, and other words to $0$.
%To accomodate imprecision in computation, we will relax this to only demand that the result by separated from 0.5 by some constant $\delta$.


%Finite/infinite precision, hardmax/softmax




\section{Regular and Context-Free Languages}\label{sec:langs} % : \textsc{Parity}

We will analyze the ability of transformers to recognize regular and context-free languages, using two prominent representatives. % (\textsc{Parity} and \textsc{2Dyck}).


\paragraph{\textsc{Parity}} is the set of bit strings such that the number of $1$s is even.
This is a very simple regular language, recognized by a finite-state automaton with two states.
The regular languages form the lowest layer of the Chomsky hierarchy, and even simple RNNs can compute all regular languages.
Within the regular languages, a particularly basic class is formed by the \emph{counter-free} or \emph{star-free} languages \cite{mcnaughton1971counter}, which can be expressed by regular expressions using only union, complementation, and concatenation.
In some sense, \textsc{Parity} is the simplest non-counter-free, or \emph{periodic}, regular language.
%Any regular language that is not counter-free whose syntactic monoid contains a group are at least computationally as hard as \textsc{Parity}.
This means, if transformers cannot compute \textsc{Parity}, they cannot recognize (almost)\footnote{Inability to compute \textsc{Parity} entails that they cannot recognize any regular language whose syntactic morphism is not quasi-aperiodic~\cite[p. 488]{barrington1992regular}.} any regular language that is not counter-free.
In the context of natural language, \textsc{Parity} naturally arises in the context of evaluating logical formulas:
%\paragraph{\textsc{LogicalFormulas}} is the set of Boolean formulas evaluating to True. Formally, we consider formulas using binary operators $\wedge, \vee$, negation $\neg$, atomic \textsc{true}, and brackets `(', `)'.
%\cite{tran2018importance} experimentally studied the ability of transformers to learn this language, finding that they did not perform as well as LSTMs.
%\textsc{LogicalFormulas} is also a natural linguistically-relevant setting in which \textsc{Parity} arises:
Evaluating iterated negations is tantamount to counting whether the number of nested negations is even or odd.
If transformers cannot compute parity, they also cannot evaluate logical formulas accurately.



%This contrastsir ability to emulate finite automata is strongly restricted, compared even to simple RNNs.


%Why is \textsc{Parity} interesting?
%The reason is that, as we will show, Parity is a very simple instantiation of certain properties that make a large range of problems hard for parallel computation with a fixed number of layers.
%If transformers cannot compute \textsc{Parity}, it immediately follows that they also cannot compute various more complex functions:




%A system that evaluates logical formulas (CITE) has to (among other things) be able to handle iterated negations.
%
%
%\cite{tran2018importance} experimentally studied the ability of transformers to learn this language, finding that they did not perform as well as LSTMs.


\paragraph{\textsc{2Dyck}} is the language of correctly bracketed words consisting of two types of brackets (`(', `)' and `[', `]').
This language is a very simple model of hierarchical structure.
The Chomsky-Sch{\"u}tzenberger theorem \cite{chomsky1963algebraic} states that any context-free language arises from a variant of \textsc{2Dyck} with multiple types of parentheses through intersection with a regular language and homomorphisms.
Consequently, the ability of LSTMs to model languages such as \textsc{2Dyck} has been an object of experimental study \cite{sennhauser2018evaluating,skachkova2018closing,bernardy2018can}.
%As \textsc{1Dyck} is a deterministic counter language, finite-precision LSTMs are capable of perfectly modeling \textsc{1Dyck};
%Empirically, though they perform less well for versions with multiple types of parentheses \cite{sennhauser2018evaluating}, which are not counter languages any more.
%In contrast, 
Our theoretical results will show that transformers are strongly limited in their ability to model \textsc{2Dyck}, including variants with fewer or more types of parentheses.


%Recognizing whether an expression is bracketed correctly is a basic example of a context-free language and of recursion. If transformers cannot compute parity, they also cannot decide whether long expressions are bracketed correctly.
%To prove this reduction, we first note that a transformer capable of deciding whether an expression is bracketed correctly can be converted into a transformer for deciding whether a string has more $1$s than $0$s. TODO actually not recognizing, but to say whether there can be a closing bracket

%Let a string be given, we then create versions $a'$ and $a''$ where $a'$ has zeroes replaced by some neutral letter, and $a''$ has its ones replaced by that.
%We then concatenate $a'a''$ and interpret zeroes as `(' and ones as `)'.
%This string can be completed to a correctly bracketed one if and only if there are at least as many `('s as there are `)'s.
%Second, we can reduce $Maj_x Maj_y (x\leq y)$ TODO unclear, might explode

%\paragraph{Finite-State Languages with Groups:} 




%\section{Finite Precision}
%We start by examining the case of finite precision.
%We interpret finite precision as meaning that only finitely many floting point numbers are distinguished.\footnote{An alternative interpretation is that numbers can be unboundedly large, but precision is limited, e.g., numbers are only distinguished in intervals of some $\epsilon$.}
%As \cite{perez2019turing} remarked, finite precision means that there are only finitely many distinct positional encodings, and the positions can be partitioned into a finite partition $\mathbb{N} = P_1 \cup \dots \cup P_N$.

%In this case, an input word as processed by the network can be reduced to listing the numbers of 0's and 1's in positions belonging to each of the subsets $P_i$.

%Such a network cannot asymptotically model $a^nb^n$.

%Such a network also cannot model \textsc{Parity}, since the finite-precision activations \dots

%show Parity is not in $Maj_2[un]$: this is actually relatively easy to show, we just have a finite number of bins, and have a numerically imprecise representation of how many are in each bin.

%This argument is quite 

%Finite precision may be an unrealistic assumption for asymptotic analysis:
%If positional embeddings have $d=100$ dimensions, they can robustly distinguish between $2^{100} \approx 10^{30}$ many positions, much more than the input length encountered in any practical NLP application.
%This motivates the study of infinite precision.





\section{Results for Hard Attention}\label{sec:hard}

We will start our analysis with the study of hard attention~\cite{perez2019turing}.
We show that hard attention transformers cannot represent \textsc{Parity} or \textsc{2Dyck}. %, or \textsc{BooleanFormula}.
To keep the results maximally general, our analysis will use combinatorial arguments and make no assumption about, e.g., activation functions and the norms of parameter matrices.
In fact, we do not even assume that the internal position-wise representations $y_{(j)}^k$ in each layer are vector-valued, as opposed to, say, discrete structures.

%\paragraph{General Formalization of Transformers}
%Each head depends on a finite number of activations in the previous layer (skip-connections) and the one argmax thing that was attended to.


%Our results will show that hard-attention transformers are not only unable to recognize these languages perfectly, they are also unable to 




\begin{figure*}[ht]
    \centering
    \begin{tabular}{cccc}
    (a) & (b) & (c) & (d) \\
    \includegraphics[width=0.23\textwidth]{figures/sa1_.png} &
        \includegraphics[width=0.23\textwidth]{figures/sa2_.png}&
    \includegraphics[width=0.22\textwidth]{figures/sa3_.png} &
        \includegraphics[width=0.23\textwidth]{figures/sa4_.png}
        \end{tabular}
	\caption{Iteratively reducing the layers of a transformer by fixing a few input symbols. (a) We fix a small number of input symbols, `attracting' attention from the first layer to a few inputs. (b) After this step, Lemma~\ref{lemma:depth-red} ensures that each activation in the first layer only depends on a small number of input symbols. (c) We again fix a few input symbols in such a way as to `attract' attention of layer-2 heads to some layer-1 activations. As a result, each layer-2 activation only depends on a small number of layer-1 activations, again by Lemma~\ref{lemma:depth-red}. (d) After this step, each layer-1 activation only depends on a few inputs, and we can remove layer 1. % In this example, input $X_5$ ends up being ignored by the transformer after applying the restriction.
	}
	\label{fig:depth-reduction}
\end{figure*}

We aim to prove that no hard-attention transformer is capable of representing \textsc{Parity} or \textsc{2Dyck}, by constructing -- for any given candidate transformer model -- a set of input words that this model will have to misclassify.
The basic idea (see Figure~\ref{fig:depth-reduction}) behind the proof is that, by fixing a small fraction of the input symbols in a particular way, we can `capture the attention' of the transformer in such a way that it ends up ignoring almost all remaining input symbols.
%More specifically, we will use the following approach.
%We take a transformer, and for each $n$, we consider what happens if we apply this transformer to inputs of that length.
%We show that, for large enough values of $n$, we can find a way to fix a small fraction of the inputs to 0 or 1, such that the prediction of the transformer depends only on a finitely bounded (independently of $n$) number of input positions.
%That is, after fixing these inputs, the transformer ignores a large number of inputs.
This shows that the transformer could not have solved a problem such as \textsc{Parity}, where every single input bit matters. %self-attention with unbounded numerical precision. %The proofs have in common the use of the probabilistic method, but the proof method in \cite{furst1984parity} does not appear to generalize to self-attention with infinite precision.


%Notably, our proof will not make any assumption about the nonlinearities, parameter matrices, positional embeddings, etc. -- all that is assumed is the basic self-attention architecture described above.
%We will not even assume that the same parameters are used for different input lengths $n$, only that the number of layers and heads is the same for all $n$.


%(On the other hand, for a problem such as AND or OR, fixing a single input is enough to fix the entire function, and transformers can do these easily indeed)
%\subsection{Preparation}
%We assume that there is only a single head per layer. If there are no restrictions on the number of layers, this is no loss of generality.
%We assume that, if two inputs have the same attention weight, the one with the smaller index is chosen. Our analysis would also work under other schemes of resolving ties, such as random selection.


In order to formalize the idea of `fixing'  a few input bits, we introduce the notion of restrictions:
A \key{restriction} $\rho$ is a family of maps $\rho_n : \{1, \dots, n\} \rightarrow \{*, 0, 1\}$ for $n \in \mathbb{N}$.
A restriction $\rho$ is applied to a transformer by restricting, when the input length is $n$, the input symbol $x_i$ to the value $\rho_n(i)$ whenever $\rho_n(i) \in \{0, 1\}$.
The output value of the resulting transformer only depends on those inputs $x_i$ such that $\rho_n(i) = *$.

The idea of using such input restrictions has been successful in the theory of Boolean circuits~\cite{furst1984parity,hastad1994optimal}.
In particular, \citet{furst1984parity}  famously used it to prove that polynomial-size bounded-depth Boolean circuits with $\wedge, \vee$, and $\neg$ gates cannot compute \textsc{Parity}.
We describe a new method to prove existence of suitable restrictions appropriate to transformers, as the proof approaches from the Boolean circuit literature do not seem to generalize to networks with real-valued activations.

The following result formalizes the claim that any transformer can be forced to ignore input bits by fixing some inputs in a particular way:
\begin{thm}\label{thm:hardmax-main}
Let any transformer be given, and let $C \in (0,1)$.
Then there is a restriction $\rho$ and an integer $c > 0$ such that 
$$|\{i \leq n: \rho_n(i) = *\}| \geq Cn$$
(for all sufficiently large $n$) and such that the function computed by the transformer on the restricted input depends only on $\leq c$ inputs, independent of input length $n$.
\end{thm}
We first show how this entails that transformers do not recognize the two formal languages:
\begin{corollary}
Transformers with hard attention cannot model \textsc{Parity} or \textsc{2Dyck}. %, or \textsc{BooleanFormula}.
%with cross-entropy asymptotically better than chance.
%, or \textsc{BooleanFormula} (i.e., asymptotic cross-entropy is at chance level).
\end{corollary}
\begin{proof}

For \textsc{Parity}, after applying a restriction, the transformer's output depends on $c$ inputs.
An input of sufficiently large size $n$ thus has unrestricted inputs that do not influence the output.
But flipping a single input bit changes the value, so the transformer's output cannot match membership in \textsc{Parity} beyond chance for such $n$.

%As a corollary, transformers also cannot model \textsc{BooleanFormula}, as they cannot model iterated negation.

For \textsc{2Dyck}, we show that hard attention transformers cannot even solve the simpler variant \textsc{1Dyck} with a single bracket type (`(', `)').
We first restrict the first $0.2n$ input positions to `(' and the last $0.2n$ positions to `)'.
After then applying the restriction provided by the theorem with $C=0.9$, the resulting restricted input will still be compatible with both well-bracketed and non-well-bracketed inputs, but the prediction will depend only on a bounded number of positions.
As the prediction depends on only a bounded number of positions, this shows the transformer could not recognize \textsc{1Dyck}, and thus also not \textsc{2Dyck}.
%The probability that 
%for BooleanFormula, maybe the reduction to PARITY is okay, if there is a constant fraction of chance that parity will be relevant
%
%TODO but cross-entropy?
\end{proof}

\paragraph{Discussion}
It may be instructive to compare to similar languages that \emph{can} be modeled by hard-attention transformers.
First, $1^*$ (over the alphabet $\{0,1\}$) is the regular language of words that have only ones and no zeroes; its minimal automaton has two states, like \textsc{Parity}.
A transformer can recognize this by having an attention head that attends to a position with zero input if it exists, and rejects if the head found such a position.
Second, $a^nb^n$ is a very basic context-free language.
It can be recognized using suitable positional embeddings by (1) having one head attend to the largest position $n$, (2) using this information to attend to any $b$ at position $<n/2$ or any $a$ at position $\geq n/2$. If such a symbol is found, the model rejects, else it accepts.
A crucial difference between these languages and \textsc{Parity} / \textsc{2Dyck} is that fixing a few inputs can easily force nonmembership, e.g. a single 0 for $1^*$, and an $a$ in the second half for $a^nb^n$.
Therefore, such simple languages are immune to the depth reduction method, and indeed \emph{can} be modeled perfectly with self-attention.

In general, the depth reduction method applies to languages that are sufficiently \emph{sensitive}: If, for some $C \in (0,1)$, fixing $Cn$ input symbols cannot force a word to be inside or outside of the language, then hard-attention transformers cannot recognize this language.
Sensitivity of functions %and the structure of Hamming neighborhoods
has been studied in computational complexity~\cite{boppana1997average,gopalan2016smooth} and more recently linked to generalization in feedforward networks~\cite{de2018deep}.
We intend to investigate these connections in future work.



\paragraph{Proof Idea of the Theorem}
Our approach for proving Theorem~\ref{thm:hardmax-main} will be to construct input restrictions in a layerwise manner, starting from layer 1. 
%Restrictions are created separately for each input length $n$.
In order for this construction to go through, the main challenge is to construct a suitable restriction at a given layer:
As shown in Figure~\ref{fig:restr}, this restriction should only affect a few input bits (about $(1-C^{1/L})n$ many input bits), while forcing each attention head in the first layer to ignore all but $c$ input bits.
Perhaps surprisingly, this is possible; the idea is to fix input bits that achieve high attention scores for several heads, so that input bits that cannot achieve such high attention scores will be ignored.



\begin{figure}[ht]
    \centering
    \begin{tabular}{cccc}
    (a) & (b) \\
    \includegraphics[width=0.21\textwidth]{writeup/figures/restr-0.png} &
        \includegraphics[width=0.21\textwidth]{writeup/figures/restr-2.png}&
        \end{tabular}
	\caption{Finding a good restriction: (a) Every attention head in the first layer could potentially attend to any input bit. (b) Perhaps surprisingly, one can fix a small number of input bits in such a way that each layer-1 attention head can only possibly attend to $c$ (here, $c=1$) inputs, and ignores all other inputs.}
	\label{fig:restr}
\end{figure}

Once we have shown that such a restriction always exists, we can use this technique to iteratively remove layers, as illustrated in Figure~\ref{fig:depth-reduction}:
After we have applied the first such restriction, each of the heads in the first layer will only depend on a bounded number $c$ of input positions.
In the second step, we apply the same argument to the heads in the second layer, so that each head in the second layer only depends on a bounded number $c'$ of heads in the first layer.
After this step, we can collapse the first layer into a collection of feedforward networks that transform a bounded number $\leq cc'$ of input positions into an activation $y_i^{(0)}$ of the lowest layer.
After this step, the first layer has been entirely removed.
Iterating this argument, we remove all layers until the prediction output only depends on a bounded number of input positions, bounded independently of input length.

We now make these ideas formal.
After the removal of the first layer of a transformer, the resulting structure is not a transformer any more, as each head in the lowest layer now depends on a \emph{combination} of input positions.
We introduce a technical definition to make this concept precise:

\begin{defin}[$c$-Transformer]
Let $c$ a positive integer. A $c$-transformer with $L$ layers is one in which the layer-0 activations $y_j^{(0)}$ depend on the embeddings not just at one position $j$, but are a function of the embeddings at $\leq c$ input positions:
\begin{equation}
    y_j^{(0)} = f^{inp}_{n,j}((v_{i_1^{j,n}}, p_{i_1^{j,n}}), \dots, (v_{i_c^{j},n}, p_{i_c^{j,n}} ))
\end{equation}
for some indices ${i_s^{j,n}} \in \{1, \dots, n\}$ ($s = 1, \dots, c$).
\end{defin}

%Note that an ordinary transformer  of depth $L$ is also a $1$-transformer of depth $L$.

With this technical notion, we show that we can reduce layers, iteratively removing the lowest layer until no self-attention layer is left:
\begin{lemma}[Depth Reduction Lemma]\label{lemma:depth-red}
Given a $c$-transformer with $L$ layers, and some restriction $\rho$ such that
\begin{equation}
|\{i \leq n: \rho_n(i) = *\}| \geq Cn
\end{equation}
($C \in (0,1]$)
for all sufficiently large $n$.
%with some restriction $\rho^0_n$ already applied, which hits a linear fraction of inputs (TODO clarify how this relates to having more heads).
Choose any $C' < C$.
Then there is a restriction $\rho'$ 
such that
\begin{equation}
|\{i \leq n: \rho'_n(i) = *\}| \geq C'n
\end{equation}
for all sufficiently large $n$, 
and such that the resulting function is computed by a $(H\cdot c\cdot(2^ck+1))$-transformer with $L-1$ layers, for some integer $k$ (depending on $C'$), where $H \geq 1$ is the number of attention heads at each layer and position.
\end{lemma}
The lemma implies Theorem~\ref{thm:hardmax-main}:
\begin{proof}[Proof of Theorem~\ref{thm:hardmax-main}]
The output of the transformer is determined by the last activation $y_{n}^{(L)}$.
Apply the Depth Reduction Lemma iteratively, choosing the constants $C'$ in the lemma appropriately, until only the zero-th layer remains.
Then, after applying the resulting restriction, the final activation $y_{n}^{(L)}$ is now computed by $y_{n}^{(0)}$, which is determined by a bounded number of input bits.
\end{proof}


%$(ab)^{2n}$ is a regular language with even numbers of a's and b's arranged in a particular order.
%Checking membership amounts to checking that (1) the length is a multiple of four, (2) the symbol is $a$ at odd positions, and $b$ at even positions.
%With suitable positional embeddings, we can construct a transformer that attends to any position violating (2)

%In particular, the resulting $c$-transformer can be rewritten into a $c'$-transformer with one layer removed.

%\begin{proof}
\subsection{Proving the Depth Reduction Lemma}
This section will be devoted to proving the Depth Reduction Lemma.
We will do the first part of the argument for all integers $k \in \mathbb{N}$.
In the second part, we will select a sufficiently large $k$. % that will work for all input lengths $n$.
%The aim is that $c' := ck$.

\paragraph{Part 1: Preliminaries}
We construct the restrictions $\rho'_n$ separately for each $n$.
Fix a layer-1 attention head $h$ ($h=1,\dots, H$) at position $i$ ($i=1, \dots, n$).
As $y^{(0)}_i$ depends on $\leq c$ input bits, it can take on at most $\leq 2^c$ possible values.
For each possible value $z$,  and each position $j \in \{1, \dots, n\}$, we compute the maximum possible attention value that can be achieved for this pair:
\begin{equation}
\max_{x_1\dots x_n\ :\ y^{(0)}_i=z} f^{att}_{1,h}(z, y^{(0)}_j)
\end{equation}
considering only inputs $x_1\dots x_n$ that are compatible with the restriction $\rho_n$.
For each value $z$, we order the positions $\{1, \dots, n\}$ downwards by this value, obtaining a sequence $j_1^{(z)}, \dots, j_n^{(z)}$ for each layer-1 attention head $h$ at a position $i$ and possible value $z$ of $y^{(0)}_i$ (In the case of ties, we order by  position, by Footnote 1).
For each layer-1 attention head and each $z$, we then select a sequence $1 \leq i_1^{(z)} < i_2^{(z)} < \dots < i_{k}^{(z)} \leq n$ such that (1) for each $i_s^{(z)}$, there is at least one input bit $x_q$ that only feeds into the activation at position $j_{i_s^{(z)}}^{(z)}$ and no $j_{i_{s'}^{(z)}}^{(z)}$ ($s\neq s'$), and (2) $i_{k}^{(z)}$ is minimal, i.e. there is no subsequence with smaller $i_{k}^{(z)}$ that also satisfies (1).
If we cannot find such a subsequence, then all $y_i^{(0)}$ together depend only on $< ck$ inputs, in which case the Depth Reduction Lemma is already satisfied for this input length $n$.

A layer-1 head $k$-\textbf{depends} on some input $x_i$ if $\rho_n(i) = *$ and $x_i$ appears as an input to some $j_r^{(z)}$ for $r \leq i_k^{(z)}$, for some $z$.
Because $i_k^{(z)}$ is minimal, a layer-1 head $k$-depends on an input if and only if that input appears as an input to some $j_{i_s^{(z)}}$ ($s \leq k$).
In particular, a layer-1 head $k$-depends only on $\leq 2^c ck$ input variables.
Two layer-1 head are $k$-\textbf{neighbors} if some $j_{i_s^{(z)}}$ for one and $j_{i_{s'}^{(z')}}$ for the other both $k$-depend on some input bit $x_l$.

We will construct input restrictions $\rho'_n$ using probabilistic arguments over random restrictions.
For this approach to succeed, we require a  sufficient amount of independence between the activations of different heads in layer 1.
We thus need to ensure that the number of $k$-neighbors of each head is bounded.
Fix some $\eta \in (0,1)$ (to be chosen later).
Let $H$ be the number of attention heads in each position of layer 1.
First, assume there is some input bit that has $>\frac{2^c}{\eta}kcH/C$ many $k$-depending layer-1 heads.
Assume the number of such input bits is larger than $\eta Cn$ for some $n$.
Then, the number of pairs of input bits and $k$-depending level-1 heads would be more than $> \eta C n \cdot \frac{2^c}{\eta} H k c/C = 2^c Hckn$ for this $n$.
But there are only $\leq 2^c Hckn$ many pairs of input bits and $k$-depending layer-1 heads -- contradiction.
Thus, the number of such input bits is at most $\leq \eta Cn$ for all $n$.
We can therefore modify the restriction $\rho_n$ so that they are set to some fixed value (doesn't matter which one) for these $<\eta Cn$ input bits, and unchanged for the others.
After this manipulation, every input bit has at most $\frac{2^c}{\eta}kcH/C$ many $k$-depending layer-1 heads, independent of $n$.

Furthermore, assume the number of input bits with $> \frac{1}{\eta} c/C$ depending layer-0 activations is $\geq \eta Cn$.
Then the number of pairs of input bits and layer-0 activations is $>\eta Cn \cdot \frac{1}{\eta} c/C = nc$.
But there are at most $nc$ such pairs, contradiction.
So the number of input bits with $> \frac{1}{\eta} c/C$ depending layer-0 heads is $\leq \eta Cn$.
We can again restrict these input bits to some fixed value. % (again, it doesn't matter which one).

After these preparations, for $n$ large enough, the set $\{i \leq n: \rho_n(i) = *\}$ has at least $(1-\eta)^2 C n$ elements.

\paragraph{Part 2: Constructing Restrictions}
After the previous part, every input bit has $\leq \frac{2^c}{\eta}kcH/C$ many $k$-depending layer-1 heads.
Consequently, every layer-1 head has at most $f \leq \frac{2^{2c}}{\eta}c^2k^2H/C$ many $k$-neighbors (for any $k$).
Also, every input bit has $\leq \frac{1}{\eta}c/C$ depending layer-0 heads.


In order to prove the existence of suitable input restrictions, we apply the ``probabilistic method'': We define a probability distribution over restrictions $\rho'_n$, and show that the probability mass assigned to restrictions of the type we require is strictly greater than zero, showing that such a restriction exists.
For each input length $n$, define the distribution over restrictions $\rho_n'$ that independently assigns to each input position $i \in \{1, \dots, n\}$ the symbol $1$ or $0$ with probability $q/2$ each ($q \in (0,1)$ to be chosen later), and $*$ with probability $1-q$.
On those input bits where $\rho_n(i) \neq *$, we restrict this random restriction to agree with $\rho_n(i)$.
For the $i$-th layer-1 attention head ($i=1,\dots,Hn$) and for each value $z$ (there are at most $2^c$ such), define $X_i^{(z)}$ to be the event that, for this head, none of $y_{j_{i_1^{(z)}}}^{(0)}, \dots, y_{j_{i_k^{(z)}}}^{(0)}$ are fixed to the value that produces the highest attention weight. % (each of these is assigned either some fixed value, , or $*$).
Define $X_0$ to be the event that a fraction greater than $(1+\delta)q$ of the input bits that $\rho_n$ maps to $*$ are set to $0/1$ by $\rho_n'$ (where $\delta \in (0,1)$, to be fixed later).
Our goal will be to show that a nonzero amount of probability mass is assigned to restrictions $\rho_n'$ avoiding all events $\{X_0\} \cup \{X_i^{(z)} : i, z\}$.

First, a Chernoff bound gives~\cite[Theorem 4.4]{mitzenmacherprobability}
%\footnote{See \url{https://en.wikipedia.org/wiki/Chernoff_bound#Multiplicative_form_(relative_error)}}
\begin{equation}
\Prob(X_0) \leq    \exp\left(-\delta^2q(1-\eta)^2Cn / 3\right)
\end{equation}
since $\rho_n$ has $\geq (1-\eta)^2Cn$ unrestricted input bits.

Second, we show that the probability of $X_i^{(z)}$ ($i=1,2,\dots, Hn$) decays exponentially in $k$.
Fixing $z$, let $Y_i^t$ ($t=1,\dots,k$) be the event that the layer-0 activation $y_{j_{i_t^{(z)}}}^{(0)}$ is not fixed to the value that produces the highest attention weight, for the given attention head $i$.
Note that $X_i^{(z)} = \bigcap_t Y_i^t$.
We have $\Prob(Y_i^s) \leq 1-(q/2)^c \in (0,1)$. 
Any $Y_i^s$ can be statistically dependent on at most $c \cdot \frac{1}{\eta}c/C = \frac{1}{\eta}c^2/C$ other events $Y_i^{s'}$, because each input bit has at most $\frac{1}{\eta} c/C$ depending layer-0 heads.
Therefore, there is a set of $\geq \frac{k}{\frac{1}{\eta}c^2/C}$ independent events among these.
Call these $Y_i^{t_1}, \dots, Y_i^{\frac{k}{\frac{1}{\eta}c^2/C}}$.
Then $X_i^{(z)} \subseteq \bigcap_{s=1}^{\frac{k}{\frac{1}{\eta}c^2/C}} Y_i^{t_s}$, and thus
\begin{equation}
\Prob(X_i^{(z)}) \leq \prod_{s=1}^{\frac{k}{\frac{1}{\eta}c^2/C}} \Prob(Y_i^{t_s}) \leq \left(1-(q/2)^c\right)^{\frac{k}{\frac{1}{\eta}c^2/C}}
\end{equation}
for each $i=1, 2, \dots, Hn$.

In order to conclude that there is a restriction $\rho'_n$ avoiding all events $\{X_0\} \cup \{X_i^{(z)} : i, z\}$, we apply the Lov{\'a}sz Local Lemma \cite[Theorem 6.17]{mitzenmacherprobability}.
%This requires that the amount of statistical dependency between these events be sufficiently `limited'.
Each event $X_i^{(z)}$ ($i=1,2,\dots, Hn$) is statistically independent of the set $\{X_j^{(z')} : \text{heads } j \text{ and } i \text{ are not $k$-neighbors}\}$.
The complement of this set has cardinality $\leq f= \frac{2^{2c}}{\eta}c^2k^2H/C$.
Set $A:=\frac{1}{k^2}$, $B:=\frac{1}{2}$.
By the Lov{\'a}sz Local Lemma, it is sufficient show the following: %, assuming $f \leq $:
\begin{align}\label{eq:lovasz-1}
&\Prob(X_i^{(z)}) \leq A(1-B)(1-A)^{f} \\ \label{eq:lovasz-2}
&\Prob(X_0)  \leq B (1-A)^{2^cHn}
\end{align}
The Lov{\'a}sz Local Lemma will then guarantee that there is some input restriction $\rho_n'$ that avoids all events $\{X_0\} \cup \{X_i^{(z)} : i, z\}$.
For~(\ref{eq:lovasz-1}), we need
\begin{align}\label{eq:x1-ineq}
    D &\leq A^{1/k}(1-B)^{1/k}(1-A)^{f/k} 
\end{align}
where $D =  \left(1-(q/2)^c\right)^{\frac{1}{\frac{1}{\eta}c^2/C}} \in (0,1)$.
For the first term on the right, 
\begin{align*}
\lim_{k\rightarrow \infty} A^{1/k} = \lim_{k\rightarrow \infty} \exp\left(-\log(k^2) / k\right) = 1
\end{align*}
Also, $\lim_{k\rightarrow \infty} (1-A)^{f/k}$ equals
\begin{align*}
\lim_{k\rightarrow \infty} \left(1-\frac{1}{k^2}\right)^{\frac{2^{2c}}{\eta}c^2kH/C} = \lim_{k\rightarrow \infty} \left(1-\frac{E^2}{k^2}\right)^{k} = 1
\end{align*}
for $E := \frac{2^{2c}}{\eta}c^2H/C$. So, if we choose $k$ large enough (independently of $n$), the RHS of (\ref{eq:x1-ineq}) can be made arbitrarily close to $1$, in particular, greater than $D$.
In order to also satisfy~(\ref{eq:lovasz-2}), we need
\begin{align*}
\exp\left(-\delta^2q(1-\eta)^2C/3\right)  \leq B^{1/n} (1-A)^{2^c H}
\end{align*}
which holds for $n$, $k$ large enough (again, choosing $k$ independent of $n$). 
In conclusion, there exists, for each sufficiently-large $n$, a restriction $\rho'_n$ that avoids all events $\{X_0\} \cup \{X_i^{(z)} : i, z\}$, for some $k$ independent of $n$.
For such a $\rho'$, we have
\begin{equation*}
|\{i \leq n: \rho_n'(i) = *\}|\geq (1-\eta)^2\cdot (1-(1+\delta)q) C n
\end{equation*}
for all sufficiently large $n$.
Then choose $\eta \in (0,1)$ small, $q \in (0,1)$ small, and $\delta >0$ (such that $(1+\delta)q \in (0,1)$) in such a way as to achieve $(1-\eta)^2\cdot (1-(1+\delta)q) = C'/C$.


After applying $\rho'_n$, every layer-1 head $b_{j,1,h}$ depends only on the $c$ input bits feeding into $y_j^{(0)}$, and the $\leq c2^ck$ input bits that the head $k$-depends on.
Thus, each layer-1 activation $y_j^{(1)}$ only depends on $\leq Hc(2^ck+1)$ input bits.
We can thus remove layer 0, convert layer-1 activations $y_j^{(1)}$ into layer-0 activations $y_j^{(0)}$, and obtain a $(Hc(2^ck+1))$-transformer performing the same computation as before when $\rho'$ is applied.
%This concludes the proof of the Depth Reduction Lemma.


%Can the same proof work for Dyck? Fix the first $0.2n$ to `(', and can we force the random restrictions to only restrict $0.3n$ (replace $2$ by $10$)?



\section{Results for Soft Attention}\label{sec:soft}

In the previous section, we showed that transformers using hard attention are not able to recognize a range of core formal languages.
In this section, we study soft attention.
It turns out that proving limitations as strong as what we found in the hard attention setting would settle a major open problem in computational complexity, and  may therefore be extremely hard to attain with currently available mathematical methods.\footnote{Showing that soft attention transformers cannot achieve perfect accuracy on evaluating Boolean formulas would separate the complexity classes $LTC^0$ and $NC^1$, a widely conjectured but long-open problem in computational complexity.}
This barrier prevents us from proving bounds on the \emph{accuracy} that soft attention transformers can achieve; nevertheless, we will be able to prove limitations on the achievable \emph{cross-entropy} in modeling distributions over the formal languages.
We will use the smoothness of the operations used in transformers to show that any transformer, as inputs get longer, will not be able to robustly model such distributions.
The idea behind the proof is that the impact of any single input symbol on the output of the transformer is small if the input is long:
\begin{lemma}\label{lem:soft-tech}
Let a soft attention transformer be given, and let $n$ be the input length.
If we exchange one input symbol $x_i$ ($i < n$), %a single input bit $x_i$ with the bit $x_i'$
then the change in the resulting activation $y_n^{(L)}$ at the decoder layer is bounded as $\mathcal{O}(\frac{1}{n})$ with constants depending on the parameter matrices.
\end{lemma}
This contrasts with recurrent networks:
Changing a single input can have nonnegligible impact on the final state even for very long input.
E.g., an RNN recognizing \textsc{Parity} through a hidden state that encodes parity of the current prefix will flip its hidden state if a single input bit is flipped.


Lemma~\ref{lem:soft-tech} entails that, as inputs become longer, soft attention transformers cannot achieve good cross-entropies on prediction problems that are very sensitive to individual input symbols:
A Lipschitz-continuous prediction function, such as a ReLU MLP with a softmax output, will not be able to make very different predictions for inputs that are encoded into similar activations $y_n^{(L)}$.

%{\textcolor{red}{TODO}}

%To make this precise, we %take a more quantitative angle and assign probability distributions over inputs for each input length $n$, and 
%consider the behavior of cross-entropy as the input length $n$ grows.
To make all our assumptions explicit, we will assume the following setting, though the results do not depend on the specific details.
For \textsc{Parity}, we consider the distribution over bitstrings generated by a two-state automaton that emits a $1$ or $0$ with probability $p/2$ each, and terminates with probability $1-p$.
Given a prefix of a string drawn from this distribution, we ask the transformer to predict the next symbol from $\Sigma = \{0, 1, \textsc{EndOfSequence}\}$.
Note that the next symbol can be \textsc{EndOfSequence} if and only if the prefix has an even number of $1$s.
For \textsc{2Dyck}, we follow the experimental study of \citet{skachkova2018closing} and take the distribution generated by a PCFG that expands $S \rightarrow SS | \epsilon$ with probability $p/3$ each, and $S  \rightarrow (S) | [S]$ with probability $p/6$ each.
We ask the model to predict the next character among $\Sigma = \{(,),[, ], \textsc{EndOfSequence}\}$. %\footnote{Only an exponentially small subset of the strings over $n$ symbols are well-labeled; thus, cross-entropy on the classification task is less meaningful for this language. Considering prediction of the next symbol sidesteps this issue.}
%We consider the problem of predicting the label from the input separately for each input length $n$, and consider cross-entropy as $n\rightarrow \infty$.

\begin{thm}
Let a soft attention transformer be given for \textsc{Parity} or \textsc{2Dyck}. %, \textsc{BooleanFormula}.
As $n\rightarrow\infty$, cross-entropy on predicting the next symbol converges to unigram chance level.
\end{thm}

\begin{proof}
For \textsc{Parity}, exchanging a single bit flips membership.
Thus, for any member of the language, there is an equally likely non-member such that output activations differ by $\mathcal{O}(\frac{1}{n})$.
Therefore, a Lipschitz-continuous prediction function cannot robustly assign different next-symbol probabilities after even and odd numbers of 1s, and cross-entropy will converge to unigram chance level.

For \textsc{2Dyck}, exchanging a single input symbol can flip whether the next symbol can be `)', `]', or only an opening bracket / \textsc{EndOfSequence}.
Thus, again, cross-entropy of a Lipschitz-continuous prediction function will converge to unigram chance level.
%, the probability that a well-bracketed prefix of length $n$ is exactly balanced is $o(1)$.
%In all other cases, the correct label includes one of the two closing parentheses, which depends on one single previous symbol, and both possibilities are equally likely if that symbol is unknown.
\end{proof}

%\subsection{Proof of Lemma~\ref{lem:soft-tech}}
%Having established the theorem, w
We proceed to proving Lemma~\ref{lem:soft-tech}.
\begin{proof}[Proof of Lemma~\ref{lem:soft-tech}]
We compare the activations at the decoder layer for two inputs that only differ in the input at the $i$-th position.
Let $D = \|v_i-v_i'\|_2$ the norm of the difference of the input embeddings at this position.
	We show by induction over $k = 1, \dots, L$ that, for some some $C > 0$ (chosen below) the differences between the resulting activations $y_j^{(k)}$, ${y_j^{(k)}}'$ are bounded as:
\begin{align*}
	\|y_i^{(k)}-{y_i^{(k)}}'\| &\leq C^{2k}D = \mathcal{O}(1) \\
	\|y_j^{(k)}-{y_j^{(k)}}'\| &\leq \frac{H^k C^{2k}D}{n} = \mathcal{O}(1/n)\ \ \ (j \neq i)
	\end{align*}
Once we have shown this, we know that the influence of any individual input on the final prediction is $\mathcal{O}(\frac{1}{n})$, with constants depending on the norms of parameter matrices and the number of layers.


For $k=0$, $\|y_i^{(0)} - {y_i^{(0)}}'\| \leq D$, %\footnote{We are assuming that $f$ is addition or concatenation \cite{vaswani2017attention}; for operations such as an MLP, there would be an additional Lipschitz constant depending on parameters and activation functions.}
and %, where $L_f$ is the Lipschitz constant of the operation $f$ combining position and input embeddings.
$\|y_j^{(0)} - {y_j^{(0)}}'\| = 0$ for $j \neq i$.
For $k>0$, we first note that for activations $\|y_j^{(k)}\|_2 \leq 2 \left(\operatorname{max}(1,L_{f^{act}})\right)^{L}  (\|p_j\| + \|v_j\|)$,
if $L_{f^{act}}$ is the Lipschitz constant of $f^{act}$.
We'll write $F$ for this upper bound for $\|y_j^{(k)}\|_2$.
As $f^{act}$ is implemented as a ReLU MLP \cite{vaswani2017attention}, $L_{f^{act}}$ depends on the norms of the parameter matrices.
Attention logits are bounded by $A := F^2 L_{f^{att}}$ in the case of multiplicative/bilinear attention, and $A := 2 F L_{f^{att}}$ in the case of additive attention.
%If attention logits are bounded as $|a_i| \leq A$,
Then any attention weight $\widehat{a}_{j,i} = \exp(a_i)/\sum_j \exp(a_j)$ is upper bounded by $\frac{\exp(A)}{\exp(A) + (n-1) \exp(-A)} \leq \frac{\exp(2A)}{n-1}$.

Choose $C : = 4(1 + 2\exp(2A) + L_{f^{act}})$. 
Recall that activations $y_i^{(k)}$ are defined as $f^{act}(y_i^{(k-1)}, b_{i,k,1}, \dots, b_{i,k,H})$, where $b_{i,k,h}$ equals $\sum_{j=1}^n \hat{a}_{i,j}^{k,h} y_j^{(k-1)}$.
We first calculate
\begin{align*}
& \|b_{j,k,h} - b_{j,k,h}'\|  \leq \sum_{w=1}^n \hat{a}_{j,w}^{k,h} \|y_w^{(k-1)} - {y}_w^{(k-1)}'\|
\\
%\end{align*}
%which is bounded as follows:
%\begin{align*}
& \leq + \hat{a}_{j,i}^{k,h} \|y_i^{(k-1)} - {y}_i^{(k-1)}'\|  + \sum_{w \neq i} \hat{a}_{j,w}^{k,h} \|y_w^{(k-1)} - {y}_w^{(k-1)}'\| 
\end{align*}
By induction hypothesis, this is bounded by:
\begin{align*}
\leq \frac{\exp(2A)}{n-1}  C^{2(k-1)} D + \frac{H^{k-1}C^{2(k-1)}D}{n} \leq \frac{H^{k-1} C^{2k-1} D}{n}
\end{align*}
Plugging this into the definition of $y_i^{(k)}$, the difference $\|y_j^{(k)} - {y_j^{(k)}}\|$ is at most
\begin{align*}
	L_{f^{act}} \cdot \left(\|y_j^{(k-1)}-{y_j^{(k-1)}}'\| + \sum_{q=1}^H \|b_{i,k,q} - b_{i,k,q}'\|\right)
\end{align*}
First, if $j= i$, this is bounded by (as $n \rightarrow \infty$)
\begin{align*}
\leq L_{f^{act}} \cdot \left(C^{2(k-1)}D + o(1)\right) \leq C^{2k}D
\end{align*}
Second, if $j\neq i$, it is bounded above by
\begin{align*}
	&  (1/n) \cdot L_{f^{act}} \cdot \left(H^{2(k-1)} C^{(k-1)}D + H^{k} C^{2k-1} D\right)
  %&   \\
\end{align*} 
which is bounded by $\leq  \frac{H^{k} C^{2k} D}{n}$.
This proves the inductive step for $k>0$.
\end{proof}




\section{Discussion}\label{sec:discussion}

We have shown that, even with infinite precision, transformers cannot robustly model non-counter-free regular languages, nor basic hierarchical structure.
In the hard attention setting, our results hold independently of activation functions and the magnitude of the parameters, and show that no transformer   can accurately classify strings as belonging to such languages.
In the soft attention setting, our results are slightly less general, but still show that transformers cannot achieve good cross-entropies when modeling distributions over these formal languages.


Our results are asymptotic in the sense that they show that any transformer will make mistakes on modeling \textsc{Parity} and \textsc{2Dyck} when the input is \emph{sufficiently long}.
A transformer may nonetheless be able to perform well on on short inputs; indeed, given any bound $N$ on the input length, it is possible to construct a transformer that will achieve perfect accuracy or cross-entropy on all examples of length $n \leq N$; our results show that the number of heads and layers, or the parameter norms, will have to increase with $N$.


How do our results compare to what is known about LSTMs?
Recurrent networks such as LSTMs can perfectly emulate finite-state automata, and therefore can model any finite state language with optimal cross-entropy, as long as the state transition and symbol emission distributions are Markovian.
In particular, \textsc{Parity} of i.i.d. bitstrings can be predicted with perfect accuracy and cross-entropy, independent of the input length.

Infinite-precision RNNs and LSTMs can model stacks \cite{tabor2000fractal,gruning2006stack,kirov2012processing} and thus are theoretically capable of modeling \textsc{2Dyck} and other deterministic context-free languages perfectly. % and \textsc{LogicalFormulas} perfectly.
This clear contrast between infinite-precision LSTMs and our findings for infinite-precision transformers may provide a theoretical explanation for the empirical finding that LSTMs seem to outperform transformers in modeling hierarchical structure \citep{tran2018importance}.


Hierarchical structure, at least at the level of context-free languages, is widely thought to be essential to modeling the structure~\cite{everaert2015structures} and meaning~\cite{montague1973proper} of natural language.
Our results entail that self-attention is limited in its ability to model context-free languages or evaluate logical formulas.
In particular, self-attention cannot in general emulate stacks or arbitrary finite-state automata.

How should we reconcile this with the success of transformers at many NLP tasks?
One possibility is that strong quantitative performance on NLP tasks can be achieved with imperfect understanding of syntactic structure, as has been suggested for other neural network models that show limited knowledge of syntax despite delivering strong perplexity  \cite{linzen2016assessing,marvin2018targeted}.
A second possibility is that practical implementations of transformers manage to circumvent such limitations by using large numbers of layers and heads, in relation to the sentence lengths typically occurring in natural language.
Relatedly, humans may also have limited capacity to solve such problems, which means that human-like language understanding may not require full computational power to solve such problems.
For instance, it has long been noted that center embeddings, syntactic structures exhibiting iterated bracketing, are very challenging for humans to process \cite{miller-finitary-1963,gibson1999memory}.
Intriguingly, self-attention bears some resemblance to psycholinguistic models of memory in human sentence processing that assume that humans, while processing a word, attend to chunks that were stored in memory when processing some previous words \cite{lewis2005activation,parker2017cue}.
Such processing models predict difficulty with center embedding because they cannot count brackets \cite{lewis2005activation}, akin to what we have shown theoretically for neural network models based on self-attention.



% Another possibility is that natural language has properties distinct from these formal languages that make it easier to process for self-attention.
%We suggest that language might avoid 
%{\color{red}
%Our results thus suggest that there might be properties of language that make transformers suitable models, despite their inability to exactly model formal languages of the types typically assumed in theoretical linguistics (e.g. they cannot do stacks).
%Maybe language due to redundancy don't have this sensitivity property
%TODO more takeaways
%-- 
%-- language doesn't have this sensitivity property
%}






\section{Conclusion}
We formally investigated the capabilities of self-attention in modeling regular languages and hierarchical structure.
We showed that transformers cannot model periodic regular languages or basic recursion, either with hard or soft attention, and even if infinite precision is allowed. %, or Boolean formula evaluation
This entails that self-attention cannot in general emulate stacks or general finite-state automata.
Our results theoretically confirm the idea that self-attention, by avoiding recurrence, has quite limited computational power.

%is not capable of modeling hierarchical structure.

%\section*{Acknowledgments}
%I thank Dan Jurafsky and the members of the Stanford NLP Group for helpful comments.

\bibliography{literature}
\bibliographystyle{acl_natbib}


\end{document}




